\documentclass[12pt, a4paper]{article}
\usepackage{sectsty}
\sectionfont{\fontsize{12}{15}\selectfont}
\subsectionfont{\fontsize{10}{12}\selectfont}
\begin{document}
\title{"Planning and Heuristic Search" \\ Oral Pre-Exam Questions (WS 2018/19)}
\date{}
\maketitle
%\input{oral-questions-1}
\section{What are nodes and edges representing in an OR graph?}
Nodes represent states of a given problem (e.g. board configurations in the 8-puzzle). Edges represent operations (solution steps) that should simplify the problem.


\section{What is a solution path in an OR-graph?}
A path \textit{P} in a state-space graph \textit{G} from node \textit{n} to goal node $\gamma$ in \textit{G}, satisfying given solution constraint,  is called a \textit{solution path} for \textit{n}. 
\subsection*{What is a solution base?}
A path \textit{P} in \textit{G} from \textit{n} to some node \textit{n'} is called \textit{solution base} for \textit{n}.


\section{What are constraint satisfaction problems? What are optimization problems?}
Constraint satisfaction problems are problems where a solution has to fulfill certain constraints and shall be found with minimum search effort. Optimization problems have to find a solution that in addition to the satisfication of constraints also has to stand out amongst all other solutions with respect to a special property.


\section{What is an appropriate representation for infinite graphs?}
The implicit representation is appropriate for infinite graphs as it uses the computable methods \textit{successors()} or \textit{next\_successor()} to determine the direct successors of a node. Its counterpart the explicit representation can only handle finite graphs, as the graph G = (V, E) is explicitly definied.


\section{What is node expansion?}
Applying the function \textit{successors(n)} on a node \textit{n} and thereby creating all direct successors of this node in \textbf{one time step} is called \textit{node expansion}. (All algorithms we considered use node expansion as a basic step, except for the backtracking algorithm)
\subsection*{What is node generation?}
Applying the function \textit{next\_successor(n)} on a node \textit{n} and thereby creating an unseen direct successor (one at a time) is called \textit{node generation}.
\subsection*{What are the states of nodes?} \begin{itemize}
\item generated (living on OPEN)
\item explored (neither on OPEN or CLOSED in A*, since it means \textit{next\_succesor} was applied and there is still at least one unseen node which will be returned by the next call of \textit{next\_succesor}.)
\item expanded (living on CLOSED, except for reopening in A*)
\item unseen
\end{itemize}



\end{document}