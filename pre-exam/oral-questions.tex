\documentclass[12pt, a4paper]{article}
\usepackage{sectsty}
\sectionfont{\fontsize{12}{15}\selectfont}
\begin{document}
\title{"Planning and Heuristic Search" \\ Oral Pre-Exam Questions (WS 2018/19)}
\date{}
\maketitle
%\input{oral-questions-1}
\section{What are nodes and edges representing in an OR graph?}
Nodes represent states of a given problem (e.g. board configurations in the 8-puzzle). Edges represent operations (solution steps) that should simplify the problem.


\section{What is a solution path in an OR-graph?}



\section{What are constraint satisfaction problems? What are optimization problems?}



\section{What is an appropriate representation for infinite graphs?}
The implicit representation is appropriate for infinite graphs as it uses the computable methods \textit{successors()} or \textit{next\_successor()} to determine the direct successors of a node. Its counterpart the explicit representation can only handle finite graphs, as the graph G = (V, E) is explicitly definied.


\section{What is node expansion?}



\end{document}